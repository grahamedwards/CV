%%%%%%%%%%%%%%%%%%%%%%%%%%%%%%%%%%%%%%%%%
% Long Curriculum Vitae
% By Graham Harper Edwards
% Adapted from http://www.rpi.edu/dept/arc/training/latex/resumes/
%%%%%%%%%%%%%%%%%%%%%%%%%%%%%%%%%%%%%%%%%

%-------------------------------------------
%	PACKAGES & CONFIGURATIONS
%-------------------------------------------

\documentclass[10pt]{article}
\usepackage{geometry}\geometry{letterpaper,margin=1in}

\usepackage{etaremune}

\usepackage{enumitem}
	\setlist[itemize]{noitemsep, topsep=0pt, leftmargin=5ex}

\usepackage{hyperref} \usepackage[dvipsnames]{xcolor} \definecolor{linkcolor}{rgb}{0.0, 0.25, 0.42} \hypersetup{colorlinks=true, urlcolor=linkcolor}
	%set link color to a lovely indigo dye.
	
\usepackage{adforn}

\usepackage{hanging} % Hanging paragraphs for minor lists.

\usepackage{fancyhdr} \usepackage{lastpage}
    \renewcommand{\headrulewidth}{0pt}
    \cfoot{G.H. Edwards \hspace{8pt}  \thepage \hspace{1pt} / \pageref*{LastPage}}
    \pagestyle{fancy}

\usepackage{titlesec}
	\titleformat*{\section}{\centering\bfseries}
	\titleformat*{\subsection}{\normalsize} %NOT \bfseries for right-aligned timeframes
	\titlespacing*{\subsection}{0ex}{2ex}{.2ex}
	\titleformat*{\subsubsection}{\normalsize\itshape}
	\titlespacing*{\subsubsection}{0ex}{1ex}{.2ex}

\renewcommand\labelitemi{\boldmath$\cdot$}

\newcommand{\ghedwards}{\textbf{G.H. Edwards}}

\setlength{\parindent}{0pt}

\begin{document}

%-------------------------------------------
%	NAME, ADDRESS, INFO
%-------------------------------------------

%Name
\begin{center}
\LARGE\textbf{\underline{Graham Harper Edwards}}\\ [.3em]
 %{\normalsize\adforn{60}}\large~~he, him~~{\normalsize\adforn{32}}
 %{\normalsize\adforn{18}}\large~~he, him~~{\normalsize\adforn{46}}
%{\normalsize\adforn{64}}\large~~he, him~~{\normalsize\adforn{36}}
\large ---~~he, him~~---
\end{center}



% Address and Info
Earth Sciences \\
Dartmouth College \hfill \textit{Email:} graham.h.edwards@dartmouth.edu \\
19 Fayerweather Hill Rd \hfill \textit{Phone:} 1~920~559~2279 \\
Hanover, NH, 03755, U.S.A.	\hfill \textit{Website:} \href{https://grahamedwards.github.io}{grahamedwards.github.io}

\vspace{2ex}

%-------------------------------------------
%	Appointments
%-------------------------------------------

\section*{CURRENT APPOINTMENT}
\subsection*{\textbf{Dartmouth College} \hfill 2021 – Present}
\textit{NSF Postdoctoral Scholar (Astronomy \&\ Astrophysics)}
\begin{itemize}
	\item[] Award \#\ 2102591
	\item[] ``Planetesimal formation, giant planet migration, and stellar accretion in our solar system and beyond"
	\item[] Advisors: Profs. C. Brenhin Keller (Earth Sciences) \&\ Elisabeth R. Newton (Physics \&\ Astronomy)
\end{itemize}

%-------------------------------------------
%	EDUCATION
%-------------------------------------------

\section*{EDUCATION}

\subsection*{\textbf{University of California Santa Cruz} \hfill 2021}
\textit{PhD}: Earth Sciences\\
\textit{Dissertation Title}: Applications of the uranium decay systems in deep time and the Quaternary: chronologic insights within planetary interiors and beneath glaciers\\
\textit{Committee}: Profs. Terrence Blackburn (Chair), Slawek Tulaczyk, Myriam Telus.

\subsection*{\textbf{Bowdoin College} \hspace{15pt} Brunswick, ME \hfill 2014}
\textit{Bachelor of Arts}: Earth and Oceanographic Science with Honors, Classics (minor) \\
\textit{Summa cum Laude}, $\Phi$BK Society

%-------------------------------------------
%	PUBLICATIONS
%-------------------------------------------

\section*{PUBLICATIONS}

\begin{etaremune} [itemsep=4pt, leftmargin=3ex]
\item \ghedwards, T. Blackburn, G. Piccione, S. Tulaczyk, G.H. Miller, C. Sikes. (\textit{in review}). Terrestrial evidence for ocean forcing of Heinrich events and subglacial hydrologic connectivity of the Laurentide Ice Sheet. \textit{Science Advances}.
\item T. Blackburn, \ghedwards, S. Tulaczyk, M. Scudder, G. Piccione, B. Hallet, J.C. Zachos, B. Cheney, N. McLean, J.T. Babbe. 2020. Ice retreat in Wilkes Basin of East Antarctica during a warm interglacial. \textit{Nature} (583), 554-559. \href{https://doi.org/10.1038/s41586-020-2484-5}{doi: 10.1038/s41586-020-2484-5}. \textbf{Press:} \href{https://www.nationalgeographic.com/science/2020/07/east-antarctic-ice-sheet-more-vulnerable-to-melting-than-thought/}{National Geographic}.
\item \ghedwards, T. Blackburn. 2020. Accretion of a large LL parent planetesimal from a recently formed chondrule population. \textit{Science Advances} (6), eaay8641. \href{https://advances.sciencemag.org/content/6/16/eaay8641}{doi: 10.1126/sciadv.aay8641}
\item \ghedwards, T. Blackburn. 2018. Detecting the extent of ca. 1.1 Ga Midcontinent Rift plume heating using U-Pb thermochronology of the lower crust. \textit{Geology} (46), 911–914. \href{https://doi.org/10.1130/G45150.1}{doi: 10.1130/G45150.1} 
\item \ghedwards. 2014. Geochemical and stratigraphic analysis of the Linnévatnet sediment record: a study of Late Holocene cirque glacier activity in Spitsbergen, Svalbard. [\textit{Honor’s Thesis}] Bowdoin College: Brunswick, ME, U.S.A. 73 pp. \href{https://digitalcommons.bowdoin.edu/honorsprojects/12/}{[Undergraduate Research Commons]}
\end{etaremune}

%-------------------------------------------
%	TEACHING
%-------------------------------------------

\section*{TEACHING}

\subsection*{\textbf{University of California Santa Cruz}}
\textit{Graduate Teaching Assistant}
\begin{tabbing} \hspace{10pt} \= \hspace{2.5cm} \=  \kill
\> Spring 2021 \> \textit{Radiogenic and Stable Isotopes}\\
\> Fall 2020 \> \textit{Evolution of Earth}\\
\> Fall 2019 \> \textit{Elements of Field Geology}\\
\> Spring 2018 \>  \textit{Geochemistry of the Solar System}\\
\> Winter 2018 \> \textit{Geologic Principles}
\end{tabbing}

\subsection*{\textbf{Bowdoin College} \hspace{15pt} Brunswick, ME, U.S.A. \hfill 2012 – 2013}
\textit{Laboratory Teaching Assistant}
\begin{tabbing} \hspace{10pt} \= \hspace{2.5cm} \=  \kill
\> Fall 2012, 2013 \> \textit{Investigating Earth}\\
\> Fall 2013 \> \textit{Marine Biogeochemistry}
\end{tabbing}

\subsection*{\textbf{Bozeman \& Livingston Public Schools} \hspace{15pt} MT, U.S.A. \hfill 2015}
\textit{Substitute Teacher}
\begin{itemize}
	\item Instructor and paraprofessional in K-12 classrooms.
	\item Taught students with various educational needs in diverse subjects.
\end{itemize}

%-------------------------------------------
%	OUTREACH SECTION
%-------------------------------------------

\section*{OUTREACH \& PUBLIC SERVICE}

\subsection*{\textbf{Montshire Museum of Science} \hfill 2022}
\textit{Guest Scientist}
\begin{itemize}
\item Lead and assist in educational activities within the museum.
\item Explain Earth and planetary science concepts to pre-K through adult audiences.
\item Develop programming focused on planetary science topics.
\end{itemize}

\subsection*{\textbf{Santa Cruz Museum of Natural History} \hfill 2018 – 2022}
\textit{Volunteer Scientist}
\begin{itemize}
	\item Created educational content and hosted activities at monthly museum events.
	\item Co-hosted monthly educational video streams covering geoscience topics. \\
	Link: \href{https://www.santacruzmuseum.org/category/rockin-pop-up/}{\emph{Rockin' Pop-Up} Archives}.
\end{itemize}

\subsection*{\textbf{Skype a Scientist} \hfill 2019 – Present}
\textit{Volunteer Scientist}
\begin{itemize}
	\item Prepare educational content and host video calls with elementary and secondary school classrooms.
	\item Teach geoscience concepts and share stories about experiences as a scientist.
\end{itemize}

\subsection*{\textbf{Geoscientists Encouraging Openness and Diversity in the Earth Sciences} \hfill 2019 – 2020}
\textit{Graduate Student Leader}
\begin{itemize}
	\item (abbreviated \textbf{GEODES}) a student-run group in the UCSC Earth \& Planetary Sciences Department.
	\item Organized and implemented outreach events centered on promoting underrepresented groups in the geosciences and cultivating a culture of inclusivity within the department.
	\item One of six graduate student leaders responsible for the operation and development of GEODES.
\end{itemize}

\subsection*{\textbf{Peary-MacMillan Arctic Museum} \hspace{15pt} Brunswick, ME, U.S.A. \hfill 2015 – 2016}
\textit{Curatorial Intern}
\begin{itemize}
	\item Developed outreach programming and exhibit content.
	\item Supplemented undergraduate courses with lectures and discussions drawing on museum collections.
	\item Mentored and trained undergraduate student interns and employees.
\end{itemize}

\subsection*{\textbf{Museum of the Rockies} \hspace{15pt} Bozeman, MT, U.S.A. \hfill 2015}
\textit{Security Volunteer}
\begin{itemize}
	\item Answered visitors' questions and facilitated interpretive discussions about natural history exhibits.
\end{itemize}

%-------------------------------------------
%	INVITED TALKS
%-------------------------------------------

\section*{INVITED TALKS}

\begin{etaremune} [itemsep=4pt, leftmargin=3ex]

   \item \textbf{Massachusetts Institute of Technology} $|$
   Planetary Lunch Seminar, 3 May 2022. \\
  Title:~\textit{The early history of the LL chondrite parent planetesimal}

  \item \textbf{Princeton University} $|$ 
  Environmental Geology and Geochemistry Seminar, 31 March 2022. \\
  Title:~\textit{Peering beneath the northern Laurentide ice sheet during the last glacial maximum}
  
\end{etaremune}
\vspace{-12pt}
%-------------------------------------------
%	ACADEMIC SERVICES, EXPERIENCES
%-------------------------------------------

\section*{ACADEMIC SERVICE \& EXPERIENCE}

\subsection*{\textbf{Mentored Undergraduates}}
\subsubsection*{Dartmouth College}
\begin{itemize}
	\item Cameron Stewart (2022)
\end{itemize}

\subsubsection*{University of California Santa Cruz}
\begin{itemize}
	\item Cosmo Varah-Sikes (2019-2021), thesis (2021)
	\item Linh Phan (2020)
	\item Michael Scudder (2018-2020), thesis (2020)
	\item Alexander Levinson (2018)
	\item Frances O'Byrne (2018)
	\item Paul Colosi (2017-2018)
\end{itemize}

\subsection*{\textbf{Departmental Service}}
\subsubsection*{Department of Earth Sciences, Dartmouth}
Earth \& Planetary History Journal Club --- Organizer 	 \hfill	Spring 2022
\begin{itemize}
	\item Scheduled, coordinated, and oversaw reading group spanning multi-disciplinary topics.
\end{itemize}

\subsubsection*{Department of Earth \& Planetary Sciences, UCSC}
Whole Earth Seminar --- Organizer \hfill  Spring 2019
\begin{itemize}
	\item One of two graduate student organizers that coordinated a departmental seminar series.
	\item Contacted and scheduled speakers from diverse disciplines in Earth and planetary sciences.
\end{itemize}

\subsection*{\textbf{Laboratory Research \& Management}}
Analytical Experience:
	\begin{itemize} [label={}]
		\item \textit{Expert:} TIMS
		\item \textit{Proficient:} MC-ICP-MS, ICP-MS, SEM-EDS
		\item \textit{Limited:} ICP-OES, XRF, XRD
	\end{itemize} \vspace{1ex}	
 \textit{W.M. Keck Isotope Facility}, UCSC \hfill 2016 -- 2021\\
Under the advisorship of lab director Prof. Terry Blackburn, I was partially responsible for the operation and maintenance of this multi-user clean lab facility, including:
	\begin{itemize} 
	\item Maintenance and operation of mass spectrometers and wet chemistry labs.
	\item Training and supporting researchers in chemical and mass spectrometric methods.
	\item Preparation, measurement, and reporting of isotopic analyses of geologic and biomineral material for external contracts.
	\item Procurement and preparation of lab consumables.
	\end{itemize}
	
\subsection*{\textbf{Conference Sessions}}
\begin{itemize}
	\item 2017 AGU Fall Meeting $|$ Co-Chair/Co-Convener: \textit{Applications of Thermochronology to Understand Crustal Systems}
\end{itemize}

\vspace{2ex}
\begin{hangparas}{5ex}{1}
	\textbf{Reviewer for:} \textit{Geochronology}, \textit{Geology}

	\vspace{2ex}
	\textbf{Society Affiliations:} Meteoritical Society, American Astronomical Society, Geochemical Society, National Association of Geoscience Teachers, Geological Society of America, American Geophysical Union.
\end{hangparas}

%-------------------------------------------
%	FUNDING
%-------------------------------------------

\section*{FUNDING}
\subsection*{\textbf{Current~/~Active}}
\begin{itemize}[leftmargin=0pt,label={},itemsep=1ex]
\item National Science Foundation --- Astronomy \&\ Astrophysics Postdoctoral Fellowship \hfill 2021 -- 2024 \\
\href{https://www.nsf.gov/awardsearch/showAward?AWD_ID=2102591&HistoricalAwards=false}{Award \#2102591}
\end{itemize}

\subsection*{\textbf{Prior~/~Inactive}}
%\begin{tabbing} \hspace{40ex} \= \hspace{25ex} \= \` \hspace{25ex}    \kill \hspace{25ex} 
\begin{itemize} [leftmargin=0pt,label={},itemsep=1ex]
\item ARCS Foundation --- Scholar Award (Fellowship) \hfill 2020 -- 2021
\item University of California Santa Cruz  --- Regent’s Fellowship \hfill 2016 – 2017
\item Bowdoin College --- Grua/O’Connell Research Award  \hfill 2013
\item Bowdoin College --- Freedman Summer Research Fellowship in Coastal/Environmental Studies \hfill 2012
\end{itemize}

%-------------------------------------------
%	AWARDS & HONORS
%-------------------------------------------

\section*{AWARDS \& HONORS}

\begin{itemize} [leftmargin=0pt,label={},itemsep=1ex]
	\item UCSC Earth \& Planetary Sciences Outstanding TA Award (Honorable Mention) \hfill 2020
	\item Aaron and Elizabeth Waters Award  \hfill 2019
	\begin{itemize} [label={}, rightmargin=30ex]
	\item \textit{Issued annually for the most outstanding proposal for PhD research in the UCSC Department of Earth \& Planetary Sciences}
	\end{itemize}
	\item National Association of Geoscience Teachers Outstanding TA Award \hfill 2014
%	\item Sarah and James Bowdoin Scholar (Dean’s List) \hfill 2011 – 2013
\end{itemize}


%-------------------------------------------
%	CONFERENCE TALK/POSTER
%-------------------------------------------

\section*{SELECTED CONFERENCE ABSTRACTS}

\begin{center} A more comprehensive list of abstracts is available on \href{https://scholar.google.com/citations?user=KHLOvgcAAAAJ&hl=en}{Google Scholar}. \end{center}

\begin{etaremune} [itemsep=4pt, leftmargin=3ex]
 \item \ghedwards, C.B. Keller, C. Stewart, E.R. Newton, 2022. A Bayesian framework for exploring the early impact history of the asteroid belt with meteorite thermochronology. \textit{85\textsuperscript{th} Annual Meeting of the Meteoritical Society}, Glasgow, Scotland. 
  \item \ghedwards, G. Piccione, M. Gomez, 2022. Comparing in-person and virtual modes of a 4-year museum-based geoscience outreach program. \textit{Earth Educators' Rendezvous 2022}, Minneapolis, MN, U.S.A.
  \item \ghedwards, T. Blackburn, G. Piccione, S. Tulaczyk, G.H. Miller, C. Sikes. 2020. [Talk]: Baffin Island subglacial precipitates record subglacial melting beneath the northern Laurentide Ice Sheet concurrent with Heinrich events. \textit{American Geophysical Union Fall Meeting}, New Orleans, LA, U.S.A.
  \item C.T. Varah-Sikes, \ghedwards, T. Blackburn. 2020. [Poster]: Thermal histories in Type 7 ordinary chondrites: interpreting residence in parent body using petrologic observation and Pb-Pb phosphate thermochronology. \textit{Geological Society of America Annual Scientific Meeting}, Virtual.
  \item \ghedwards, T. Blackburn, S. Tulaczyk, G.G. Piccione. 2019. [Poster]: U-series isotopics of silts from Taylor Valley, Antarctica: Applying comminution dating methods on polar desert tills and implications for the timeframe of valley incision. \textit{American Geophysical Union Fall Meeting}, San Francisco, CA, U.S.A.
  \item  \ghedwards, T. Blackburn, S. Tulaczyk, G.G. Piccione. 2019. [Poster]: U-Series isotopics constrain timescale of bedrock comminution and glacial incision in Taylor Valley, Antarctica. \textit{Goldschmidt Annual Meeting}, Barcelona, ES.
	\item  \ghedwards, T. Blackburn, G.G. Piccione. 2018. [Talk]: Cooling and disruption of the LL chondrite parent body. \textit{Lunar and Small Bodies Graduate Forum (LunGradCon)}, Mountain View, CA, U.S.A.
  \item  \ghedwards, T. Blackburn, C.M.O’D. Alexander. 2017. [Talk]: Accretion and disruption histories of the ordinary chondrite parent bodies. \textit{Meteoritical Society Annual Meeting}, Santa Fe, NM, U.S.A.
  \item \ghedwards, T. Blackburn, K.V. Smit. 2017. [Poster]: Timescales of crustal cooling of the Superior Craton near Attawapiskat, Ontario, Canada, and implications for extent of Keweenawan plume heating. \textit{American Geophysical Union Fall Meeting}, New Orleans, LA, U.S.A.
  \item \ghedwards. 2014. [Poster]: Geochemical and stratigraphic analysis of the Linnévatnet sediment record: a provenance study of Late Holocene cirque glacier activity in Linnédalen, Spitsbergen, Svalbard. \textit{44th International Arctic Workshop}, Boulder, CO, U.S.A.
\end{etaremune}
\vspace{-12pt}

%-----------------------------------------------------
% 	ADDITIONAL WORK EXPERIENCE
%-----------------------------------------------------
\begin{center}{\rule{2.5in}{0.5pt}}\end{center}  \begin{center}\vspace{-12pt} {\rule{2.5in}{0.5pt}}\end{center}
	
\subsection*{\textbf{Additional Work Experience}}
\begin{itemize}[label={}]
	\item Server and Bartender --- Trio Restaurant, Fish Creek, WI, U.S.A. --- 2014.
	\item Wildland Firefighter (Type 2) --- U.S. Forest Service, Winthrop, WA, U.S.A. --- 2015.
\end{itemize}


\subsection*{\textbf{Non-Academic Interests}} \hspace{5ex}
 Rock climbing, cross-country skiing, birdwatching, breadmaking, classical history and literature.

%-------------------------------------------
%-------------------------------------------
\end{document}
