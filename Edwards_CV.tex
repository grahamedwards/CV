%%%%%%%%%%%%%%%%%%%%%%%%%%%%%%%%%%%%%%%%%
% Long Curriculum Vitae
% By Graham Harper Edwards
% Adapted from http://www.rpi.edu/dept/arc/training/latex/resumes/
%%%%%%%%%%%%%%%%%%%%%%%%%%%%%%%%%%%%%%%%%

%-------------------------------------------
%	PACKAGES & CONFIGURATIONS
%-------------------------------------------

\documentclass[10pt]{article}
\usepackage{geometry}\geometry{letterpaper,margin=1in}

\usepackage{etaremune}

\usepackage{enumitem}
	\setlist[itemize]{noitemsep, topsep=0pt, leftmargin=5ex}

\usepackage{hyperref} \usepackage[dvipsnames]{xcolor} \definecolor{linkcolor}{rgb}{0.0, 0.25, 0.42} \hypersetup{colorlinks=true, urlcolor=linkcolor}
	%set link color to a lovely indigo dye.

\usepackage{hanging} % Hanging paragraphs for minor lists.

\usepackage{fancyhdr} \usepackage{lastpage}
    \renewcommand{\headrulewidth}{0pt}
    \cfoot{G.H. Edwards \hspace{8pt}  \thepage \hspace{1pt} / \pageref*{LastPage}}
    \pagestyle{fancy}

\usepackage{titlesec}
	\titleformat*{\section}{\centering\bfseries}
	\titleformat*{\subsection}{\normalsize}
	\titlespacing*{\subsection}{0ex}{2ex}{.2ex}

\renewcommand\labelitemi{\boldmath$\cdot$}

\setlength{\parindent}{0pt}

\begin{document}

%-------------------------------------------
%	NAME, ADDRESS, INFO
%-------------------------------------------

%Name
\begingroup
\centering \LARGE \textbf{\underline{Graham Harper Edwards}} \\ [1em]
\endgroup

% Address and Info
Earth \& Planetary Sciences \\
University of California Santa Cruz \hfill \textit{Email:} ghedwards@ucsc.edu \\
1156 High Street EMS A232 \hfill \textit{Phone:} 1.920.559.2279 \\
Santa Cruz, CA 95060 U.S.A.	\hfill \textit{Website:} \href{https://grahamedwards.github.io}{grahamedwards.github.io}

\vspace{2ex}
%-------------------------------------------
%	STATEMENT
%-------------------------------------------
\iffalse
\section{STATEMENT}

\textbf{Research} \ I am a geochemist and geochronologist: my primary research interest is to resolve and refine timescales of Earth and solar system processes via radiometric records. My doctoral research has focused on the application of the uranium decay systems to uncover the tempos of high-temperature processes in deep time and subglacial processes in the Quaternary. In my future work, I am excited to hone my statistical techniques to apply to database studies of meteorite chemistries.

In addition to my research interests, I am dually committed to the inclusion, support, and retention of individuals and groups of diverse identities and backgrounds.
\fi

%-------------------------------------------
%	EDUCATION
%-------------------------------------------

\section*{EDUCATION}

\subsection*{\textbf{University of California Santa Cruz} \hfill 2016 – Present}
\textit{Doctoral Studies}: Earth \& Planetary Sciences\\
Candidacy Fall 2018 \\
\textit{Proposed Dissertation Title}: Applications of the uranium decay chain to diverse problems in deep time and Quaternary studies: chronologic insights into processes of planetary interiors and the bases of glaciers.\\
\textit{Advisor}: Prof. Terrence Blackburn\\
\textit{Committee}: Profs. Slawek Tulaczyk ({\sl Chair}, UCSC), Myriam Telus (UCSC), Kurt Cuffey (UC Berkeley)

\subsection*{\textbf{Bowdoin College} \hspace{15pt} Brunswick, ME \hfill 2014}
\textit{Bachelor of Arts}: Earth and Oceanographic Science with Honors, Classics minor \\
\textit{Summa cum Laude, $\Phi$BK Society}

%-------------------------------------------
%	PUBLICATIONS
%-------------------------------------------

\section*{PUBLICATIONS}

\begin{etaremune} [itemsep=4pt, leftmargin=3ex]
  \item T. Blackburn, G.H. Edwards, S. Tulaczyk, M. Scudder, G. Piccione, B. Hallet, J.C. Zachos, B. Cheney, N. McLean, J.T. Babbe. 2020. Ice retreat in Wilkes Basin of East Antarctica during a warm interglacial. \textit{Nature} (583), 554-559. \href{https://doi.org/10.1038/s41586-020-2484-5}{doi: 10.1038/s41586-020-2484-5}. \textbf{Press:} \href{https://www.nationalgeographic.com/science/2020/07/east-antarctic-ice-sheet-more-vulnerable-to-melting-than-thought/}{National Geographic}.
  \item G.H. Edwards, T. Blackburn. 2020. Accretion of a large LL parent planetesimal from a recently formed chondrule population. \textit{Science Advances} (6), eaay8641. \href{https://advances.sciencemag.org/content/6/16/eaay8641}{doi: 10.1126/sciadv.aay8641}
  \item G.H. Edwards, T. Blackburn. 2018. Detecting the extent of ca. 1.1 Ga Midcontinent Rift plume heating using U-Pb thermochronology of the lower crust. \textit{Geology} (46), 911–914. \href{https://doi.org/10.1130/G45150.1}{doi: 10.1130/G45150.1}
  \item G.H. Edwards. 2014. Geochemical and stratigraphic analysis of the Linnévatnet sediment record: a study of Late Holocene cirque glacier activity in Spitsbergen, Svalbard. [\textit{Honor’s Thesis}] Bowdoin College: Brunswick, ME, U.S.A. 73 pp. \href{https://digitalcommons.bowdoin.edu/honorsprojects/12/}{[Undergraduate Research Commons]}
\end{etaremune}

%-------------------------------------------
%	TEACHING
%-------------------------------------------

\section*{TEACHING}

\subsection*{\textbf{University of California Santa Cruz}}
\textit{Graduate Teaching Assistant}
\begin{tabbing} \hspace{10pt} \= \hspace{2.5cm} \=  \kill

\> Fall 2020 \> \textit{Evolution of Earth}\\
\> Fall 2019 \> \textit{Elements of Field Geology}\\
\> Spring 2018 \>  \textit{Geochemistry of the Solar System}\\
\> Winter 2018 \> \textit{Geologic Principles}

\end{tabbing}

\subsection*{\textbf{Bozeman \& Livingston Public Schools} \hspace{15pt} MT, U.S.A. \hfill 2015}
\textit{Substitute Teacher}
\begin{itemize}
	\item Instructor and paraprofessional in K-12 classrooms.
	\item Taught students with various educational needs in diverse subjects.
\end{itemize}

\subsection*{\textbf{Bowdoin College} \hspace{15pt} Brunswick, ME, U.S.A. \hfill 2012 – 2013}
\textit{Laboratory Teaching Assistant}
\begin{itemize}
	\item Assisted instruction of field and laboratory activities for an introductory geoscience course.
\end{itemize}

%-------------------------------------------
%	OUTREACH SECTION
%-------------------------------------------

\section*{OUTREACH \& PUBLIC SERVICE}

\subsection*{\textbf{Santa Cruz Museum of Natural History} \hfill 2018 – Present}
\textit{Volunteer Scientist}
\begin{itemize}
	\item Create educational content and host geoscience educational activities at monthly museum events.
	\item During covid-19 pandemic shelter-in-place orders, prepared and performed in weekly educational video streams covering geoscience topics. Archived \emph{Rockin' Pop-Up} videos are available \href{https://www.santacruzmuseum.org/category/rockin-pop-up/}{here}.
\end{itemize}

\subsection*{\textbf{Geoscientists Encouraging Openness and Diversity in the Earth Sciences} \hfill 2019 – 2020}
\textbf{(GEODES)} \hspace{3pt} \textit{Student Leader}
\begin{itemize}
	\item Organize and implement outreach events centered on promoting awareness of issues facing underrepresented groups in the geosciences and cultivating a culture of inclusivity in the UCSC Earth \& Planetary Sciences Dept.
	\item One of six graduate student leaders responsible for the operation and development of GEODES.
\end{itemize}

\subsection*{\textbf{Skype a Scientist} \hfill 2019 – 2020}
\textit{Volunteer Scientist}
\begin{itemize}
	\item Prepare educational content and host video calls with elementary classrooms.
	\item Teach geoscience concepts and share stories about experiences as a scientist.
\end{itemize}

\subsection*{\textbf{Peary-MacMillan Arctic Museum} \hspace{15pt} Brunswick, ME, U.S.A. \hfill 2015 – 2016}
\textit{Curatorial Intern}
\begin{itemize}
	\item Developed outreach programming and exhibition content.
	\item Supplemented undergraduate courses with lectures drawing on museum collections.
	\item Mentored and trained undergraduate student interns and employees.
\end{itemize}

\subsection*{\textbf{Museum of the Rockies} \hspace{15pt} Bozeman, MT, U.S.A. \hfill 2015}
\textit{Security Volunteer}
\begin{itemize}
	\item Answered visitors' questions and facilitating interpretive discussions about natural history exhibits.
	\item Enforced museum rules as an introductory volunteer (docents required multi-year training).
\end{itemize}

%-------------------------------------------
%	AWARDS & HONORS
%-------------------------------------------

\section*{AWARDS \& HONORS}

\begin{itemize} [leftmargin=0pt,label={},itemsep=1ex]
	\item ARCS Foundation Scholar Award (Fellowship) \hfill 2020 – 2021
	\item UCSC Earth \& Planetary Sciences Outstanding TA Award (Honorable Mention) \hfill 2020
	\item Aaron and Elizabeth Waters Award  \hfill 2019
	\begin{itemize} [label={}, rightmargin=30ex]
	\item \textit{Issued annually for the most outstanding proposal for PhD research in the UCSC Earth \& Planetary Sciences Dept.}
	\end{itemize}
	\item UC Santa Cruz Regent’s Fellowship \hfill 2016 – 2017
	\item National Association of Geoscience Teachers Outstanding TA Award \hfill 2014
	\item Sarah and James Bowdoin Scholar (Dean’s List) \hfill 2011 – 2013
	\item Grua/O’Connell Research Award  \hfill 2013
	\item Freedman Summer Research Fellowship in Coastal/Environmental Studies \hfill 2012
\end{itemize}

%-------------------------------------------
%	CONFERENCE TALK/POSTER
%-------------------------------------------

\section*{CONFERENCE ABSTRACTS (1\textsuperscript{\tiny{ST}} AUTHOR)}

\begin{center} A more comprehensive list of co-authored abstracts is available on \href{https://scholar.google.com/citations?user=KHLOvgcAAAAJ&hl=en}{Google Scholar}. \end{center}

\begin{etaremune} [itemsep=4pt, leftmargin=3ex]
  \item G.H. Edwards, T. Blackburn, S. Tulaczyk, G.G. Piccione. 2019. [Poster]: U-series isotopics of silts from Taylor Valley, Antarctica: Applying comminution dating methods on polar desert tills and implications for the timeframe of valley incision. \textit{American Geophysical Union Fall Meeting}, San Francisco, CA, U.S.A.
  \item  G.H. Edwards, T. Blackburn, S. Tulaczyk, G.G. Piccione. 2019. [Poster]: U-Series isotopics constrain timescale of bedrock comminution and glacial incision in Taylor Valley, Antarctica. \textit{Goldschmidt Annual Meeting}, Barcelona, ES.
	\item  G.H. Edwards, T. Blackburn, G.G. Piccione. 2018. [Talk]: Cooling and disruption of the LL chondrite parent body. \textit{Lunar and Small Bodies Graduate Forum (LunGradCon)}, Mountain View, CA, U.S.A.
  \item  G.H. Edwards, T. Blackburn, C.M.O’D. Alexander. 2017. [Talk]: Accretion and disruption histories of the ordinary chondrite parent bodies. \textit{Meteoritical Society Annual Meeting}, Santa Fe, NM, U.S.A.
  \item G.H. Edwards, T. Blackburn, K.V. Smit. 2017. [Poster]: Timescales of crustal cooling of the Superior Craton near Attawapiskat, Ontario, Canada, and implications for extent of Keweenawan plume heating. \textit{American Geophysical Union Fall Meeting}, New Orleans, LA, U.S.A.
  \item G.H. Edwards. 2014. [Poster]: Geochemical and stratigraphic analysis of the Linnévatnet sediment record: a provenance study of Late Holocene cirque glacier activity in Linnédalen, Spitsbergen, Svalbard. \textit{44th International Arctic Workshop}, Boulder, CO, U.S.A.
\end{etaremune}
\vspace{-12pt}

%-------------------------------------------
%	ADDITIONAL SERVICES, EXPERIENCES
%-------------------------------------------

\section*{ACADEMIC SERVICE, EXPERIENCE}

\subsection*{\textbf{Mentored Undergraduates}}
\begin{itemize}
		\item Cosmo Varah-Sikes, (2019-2020), planned thesis (2021)
		\item Linh Phan (2020)
		\item Michael Scudder (2018-2020), thesis (2020)
		\item Alexander Levinson (2018)
		\item Paul Colosi (2017-2018)
\end{itemize}

\subsection*{\textbf{Conference Sessions}}
\begin{itemize}
	\item 2017 AGU Fall Meeting $|$ Co-Chair/Co-Convener: \textit{Applications of Thermochronology to Understand Crustal Systems}
\end{itemize}

\subsection*{\textbf{Seminars and Talks}}
\textit{Whole Earth Seminar Organizer}, Department of Earth \& Planetary Sciences, UCSC \hfill  Spring 2019
\begin{itemize}
	\item One of two graduate student organizers that coordinated a departmental seminar series.
	\item Contacted and scheduled speakers from diverse disciplines in Earth and planetary sciences.
\end{itemize}

\vspace{2ex}
\begin{hangparas}{5ex}{1}
	\textbf{Reviewer for:} \textit{Geochronology}

	\vspace{2ex}
	\textbf{Society Affiliations:} Geochemical Society, Geological Society of America, Meteoritical Society, American Geophysical Union.
\end{hangparas}

\begin{center}{\rule{2.5in}{0.5pt}}\end{center}

\subsection*{\textbf{Analytical Experience}}
\begin{itemize} [label={}]
	\item \textit{Expert:} TIMS
	\item \textit{Proficient:} MC-ICP-MS, ICP-MS, SEM-EDS
	\item \textit{Limited:} ICP-OES, XRF, XRD
\end{itemize}


\subsection*{\textbf{Additional Work Experience}}
\begin{itemize}[label={}]
	\item Server and Bartender --- Trio Restaurant, Fish Creek, WI --- 2014.
	\item Wildland Firefighter --- USFS handcrew, Winthrop, WA --- 2015.
\end{itemize}


\subsection*{\textbf{Non-Academic Interests}} \hspace{5ex}
 Rock climbing, cross-country skiing, birdwatching, breadmaking, classical history and literature.

%-------------------------------------------
%-------------------------------------------
\end{document}
