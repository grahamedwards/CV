\section{PUBLICATIONS}
\hfill* mentored student author
\newcommand{\inprep}[1]{\textit{#1} [in prep.]}
\newcommand{\submitted}[1]{\textit{#1} [submitted]}
\newcommand{\inreview}[1]{\textit{#1} [in review]}
\newcommand{\inpress}[1]{\textit{#1} [in press]}
\newcommand{\doi}[1]{doi:\href{https://doi.org/#1}{\texttt{#1}}}
\newcommand{\arxiv}[1]{arXiv:\href{https://arxiv.org/abs/#1}{\texttt{#1}}}
\newcommand{\press}[2]{\textbf{Press:}~\href{#1}{#2}}
\newcommand{\pub}[2]{\textit{#1},~\textbf{#2}}

\lineitem{Refereed}
\begin{etaremune} [itemsep=4pt, leftmargin=3ex]
    %\item[\dots] \ghedwards, G. Piccione, M. Gomez. Viewership patterns of a virtual geoscience outreach program. \inprep{Journal of Geoscience Education}.
    \item R.~Rampalli*, J.W.~Johnson, M.K.~Ness, \ghedwards, E.R.~Newton, E.J.~Griffith, M.~Bedell, K.~Wang. 2025. A galactic perspective on the (unremarkable) relative refractory depletion observed in the Sun. \inreview{The Astrophysical Journal}. \arxiv{2509.03577}.

    \item A.~Abuawad, M.~Griffiths, \ghedwards, A.~Eftekhari, M.~El-Ebweini, H.~Al-Najar, A.~Butmeh, R.~Abu~Dayyeh, M.~El-Shewy, A.~Aker. 2025. The ongoing environmental destruction and degradation of Gaza: the resulting public health crisis. \pub{American Journal of Public Health}{115}. \\\doi{10.2105/AJPH.2025.308140}. SSRN preprint:~\doi{10.2139/ssrn.5021472}.\TUpubpct{10}
 
    \item \ghedwards, G.G.~Piccione, T.~Blackburn,  S.~Tulaczyk. 2025. Uranium-series isotopes as tracers of physical and chemical weathering in glacial sediments from Taylor Valley, Antarctica. \pub{Chemical Geology}{671}. \doi{10.1016/j.chemgeo.2024.122463}.\TUpubpct{70}
        
    \item \ghedwards, C.B.~Keller, E.R.~Newton, C.W.~Stewart*. 2024. An early giant planet instability recorded in asteroidal meteorites. \pub{Nature Astronomy}{8}. \doi{10.1038/s41550-024-02340-6}. \\\arxiv{2309.10906}. \press{https://astrobites.org/2023/09/25/meteorites-planet-migration/}{Astrobites}.\TUpubpct{90}
    
    \item \ghedwards. 2024. Giant planets migrated shortly after the Solar System’s protoplanetary disk dispersed. \textit{Nature Astronomy Research Briefing}. \doi{10.1038/s41550-024-02341-5}.\TUpubpct{100}
    
    \item R.~Rampalli*, M.K.~Ness, \ghedwards, E.R.~Newton, M.~Bedell. 2024. The Sun remains relatively refractory depleted: elemental abundances for $17,412$ Gaia RVS solar analogs and 50 planet hosts. \pub{The Astrophysical Journal}{965}. \doi{10.3847/1538-4357/ad303e}. \arxiv{2402.16954}.
    
    \item M.A.~Thompson, M.~Telus, \ghedwards, et al. 2023. Outgassing composition of the Murchison meteorite: implications for volatile depletion of planetesimals and interior-atmosphere connections for terrestrial exoplanets. \pub{Planetary Science Journal}{4}. \doi{10.3847/PSJ/acf760}. \arxiv{2310.02028}.
    
    \item \ghedwards, T. Blackburn, G. Piccione, S. Tulaczyk, G.H. Miller, C. Sikes*. 2022. Terrestrial evidence for ocean forcing of Heinrich events and subglacial hydrologic connectivity of the Laurentide Ice Sheet. \pub{Science Advances}{8}. \doi{10.1126/sciadv.abp9329}.
    
    \item T. Blackburn, \ghedwards, S. Tulaczyk, M. Scudder*, G. Piccione, B. Hallet, J.C. Zachos, B. Cheney, N. McLean, J.T. Babbe. 2020. Ice retreat in Wilkes Basin of East Antarctica during a warm interglacial. \pub{Nature}{583}. \doi{10.1038/s41586-020-2484-5}. \press{https://www.nationalgeographic.com/science/2020/07/east-antarctic-ice-sheet-more-vulnerable-to-melting-than-thought/}{National Geographic}.
    
    \item \ghedwards, T. Blackburn. 2020. Accretion of a large LL parent planetesimal from a recently formed chondrule population. \pub{Science Advances}{6}. \doi{10.1126/sciadv.aay8641}.
    
    \item \ghedwards, T. Blackburn. 2018. Detecting the extent of ca. 1.1 Ga Midcontinent Rift plume heating using U-Pb thermochronology of the lower crust. \pub{Geology}{46}. \doi{10.1130/G45150.1}.
    


\end{etaremune}

\lineitem{Non-refereed}
\begin{etaremune} [itemsep=4pt, leftmargin=3ex]
    \item \ghedwards. 2014. Geochemical and stratigraphic analysis of the Linnévatnet sediment record: a study of Late Holocene cirque glacier activity in Spitsbergen, Svalbard. [\textit{Honor’s Thesis}] Bowdoin College: Brunswick, ME, U.S.A. 73 pp. [\href{https://digitalcommons.bowdoin.edu/honorsprojects/12/}{Undergraduate Research Commons}]
\end{etaremune}
